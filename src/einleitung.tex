\section{Einleitung}\label{sec:einleitung}
Der Mensch ist sehr gut darin Objekte, Motive und Zeichen zu erkennen, die auf einem Bild abgebildet sind.
Das menschliche Gehirn kann dabei auch Bildern die gleiche Bezeichnung zuordnen, die sich von allen anderen Bildern unterscheiden, die es jemals gesehen hat.
Beispielsweise können Menschen eine fremde Handschrift lesen, obwohl sie diese zuvor noch nie gesehen haben.
Ein Computer kann dagegen nur festen Programmabläufen folgen und ist allein mit vorbestimmten Algorithmen nicht in der Lage, unbekannte Daten zu analysieren und zu interpretieren.
Künstliche neuronale Netze versuchen, dem Computer das Prinzip des Lernens beizubringen.
Auf dieses Prinzip wird in dieser Arbeit eingegangen, indem erst der Aufbau und die Funktionsweise eines solchen künstlichen neuronalen Netzes erklärt werden.
Anschließend geht es um die dadurch gewonnene Lernfähigkeit und dessen Anwendung.
Das Ziel dieser Arbeit ist es, die Theorie und Anwendung von einfachen künstlichen neuronalen Netzen zu erklären und an einem Beispiel der Bilderkennung zu zeigen.\\
Künstliche Intelligenz ist für die meisten Menschen schon ein Teil ihres Alltags geworden, ohne dass sie es gemerkt haben.
Von der Autokorrektur der Handytastatur bis hin zu Sprach-, Bild- und Gesichtserkennung funktioniert heutzutage vieles mit künstlicher Intelligenz und künstlichen neuronalen Netzen.
Inzwischen setzen viele Menschen auf die Technik, die durch die neuronalen Netze ermöglicht wird.
Diese vielen verschiedenen Anwendungsmöglichkeiten und deren mathematischer Hintergrund sind der Grund für mein Interesse an diesem Thema.